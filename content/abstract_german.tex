\thispagestyle{empty}
\begin{center}\textsf{\textbf{Zusammenfassung}}\end{center}

\noindent Eine zentrale Herausforderung der Softwareentwicklung stellt die Parallelisierung von Algorithmen aus Gründen der Performancesteigerung dar. Das Erzeugen von parallelen Programmen erfordert tiefgründige Kenntnis der verfügbaren Hardwarearchitekturen und nutzbaren Programmiermodelle. Um eine gehobene Form der Parallelisierung zu erreichen, ist es außerdem notwendig, Programme mit Multi-Core-Unterstützung auf mehrere Maschinen innerhalb eines Clusters zu verteilen. Dieser Schritt führt zu einem erhöhten Kommunikationsaufwand, der die Komplexität der jeweiligen Algorithmen erhöht. Das Forschungsziel dieser Arbeit ist die Schaffung eines Frameworks zur Verteilung von parallelen Berechnungen auf CPUs und GPUs innerhalb eines Clusters, welches seine Funktionalität über eine API auf hoher Abstraktionsebene bereitstellt. Somit sollen Nutzer ermächtigt werden, ohne detaillierte Fachkenntnis verteilte parallele Programme zu erstellen. Als weiteres Hauptmerkmal soll es möglich sein, die Ressourcen eines Clusters dynamisch an die benötigten Rechenkapazitäten anzupassen.

Aufgrund der Vielzahl an verfügbaren Parallelisierungsmethoden wie OpenMP, MPI und OpenCL ist es notwendig, das jeweilige Nutzungspotenzial für das Framework zu bestimmen. Da OpenCL es ermöglicht, portable parallele Programme über ein festes Programmiermodell zu erzeugen, wird es als zentrale Technologie des Frameworks ausgewählt. In Kombination mit dOpenCL können OpenCL Programme ohne Änderungen des Programmcodes auf entfernten Maschinen ausgeführt werden. Das daraus resultierende Parallelisierungspotenzial wird innerhalb des implementierten Frameworks, genannt Dynamic OpenCL, abstrahiert. Es stellt somit den zentralen Beitrag dieser Forschung dar. Dynamic OpenCL ist in Java verfasst und ermöglicht es Programmierern, verteilte OpenCL Programme in Java zu erzeugen. Durch Mechanismen zur Verwaltung individueller Clusterstrukturen und über die Möglichkeit der dynamischen Anpassung der verwendeten Ressourcen kann Dynamic OpenCL in unterschiedlichen Anwendungsfällen eingesetzt werden.

Während der ausführlichen Evaluierung von Dynamic OpenCL kann gezeigt werden, dass erhebliche Performancesteigerungen durch die Parallelisierung möglich sind. Auch in komplexen Clusterstrukturen mit angebundenen Cloudressourcen können Ausführungszeiten mit Hilfe von spezialisierten Scheduling-Algorithmen signifikant verringert werden. Auf der Basis von Dynamic OpenCL wird ein prototypischer Web-Server präsentiert, der Nutzern die Ausführung von vorgefertigten Algorithmen über ein grafische Oberfläche ermöglicht. Die Oberfläche erlaubt es ebenfalls, die Rechenkapazität des Clusters dynamisch nach Belieben anzupassen.

Es wird aufgezeigt, dass Dynamic OpenCL in der Lage ist, in unterschiedlichen Umgebungen deutliche Leistungssteigerungen herbeizuführen. Dabei ist das Potenzial der Verbesserung größtenteils abhängig vom jeweiligen Algorithmus und der verfügbaren Netzwerkbandbreite des Clusters. Durch zukünftige Verbesserungen an Dynamic OpenCL, die Engpässe bei der Verteilung von Aufgaben über das Netzwerk umgehen, können weitere Beschleunigungen in Aussicht gestellt werden.