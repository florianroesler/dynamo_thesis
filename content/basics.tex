\chapter{Basics}

\section{Distribution Methods}
\label{distribution_basics}

\subsection{OpenMP and OpenACC}
OpenMP represents a compiler extension that allows programmers to harness parallel computational power on a machine. It adds functions and flags to the respective language (C, C++ and Fortran are currently supported), which indicate the execution environment how to parallelize the program \cite{openmp_spec}. The resulting code can then be run in multiple threads that can be scheduled on many CPU cores in parallel.

\begin{lstlisting}[language=C]
void example()
{
int i = 0;
int n = 10;
#pragma omp parallel for
for (i=1; i<n; i++) /* i is private by default */
b[i] = (a[i] + a[i-1]) / 2.0;
}

\end{lstlisting}


OpenACC offers similar functionality as OpenMP but aims at offloading parallel computations to attached accelerators like GPUs \cite{openacc_spec}. 

\subsection{MPI}
Although OpenMP and OpenACC are powerful tools for parallelization on a single machine, in order to write large scale software, communications across multiple machines are necessary. One standard tool to achieve this is MPI.
MPI offers bindings to many languages and extends such by functions to identify a process and send as well as receive messages.

\subsection{MapReduce}
Due to its simple programming model, which mainly consists of a \textit{Map} and a \textit{Reduce} phase it is a commonly used approach for large clusters. Its most prominent implementation is Hadoop MapReduce, which runs in combination with the Hadoop Distributed File System. It is therefore especially applicable for data intense jobs like log analysis.

\section{CUDA}

CUDA is an API released by NVIDIA, which enables NVIDIA GPUs to be used for general purpose computing. As modern GPUs are comprised of hundreds or even thousands of streaming processors, data parallel algorithms can make extensive use of the CUDA infrastructure.

\section{OpenCL}
\label{opencl}
OpenCL is a framework that represents a direct competitor to CUDA. It was initially proposed by Apple but is now currently by the Khronos Group, which consists of a multitude of companies like AMD, Intel, NVIDIA and others \cite{opencl_directors}. Unlike CUDA, OpenCL is not limited to NVIDIA GPUs but supports various device types and vendors. Instead OpenCL programs may be executed on CPUs, GPUs, FPGAs and digital signal processor (abbr. DSP). Prerequisite for the execution of OpenCL code on a machine is the presence of an OpenCL Installable Client Driver (abbr. ICD) for a targeted device. Such ICDs must conform with the specifications released by the Khronos Group. As of today, many implementations of the specifications exist by various vendors like AMD, NVIDIA, Intel, Apple, ARM, Xilinx and others \cite{opencl_conform}.

The most important definitions within the OpenCL standard are the introduction of its programming model as well as the underlying memory model:

\begin{description}[style=nextline]
    \item [Programming Model]
    OpenCL programs, called Kernels are executed in parallel on devices, namely for each work-item, which is the smallest partial unit of a submitted task. For example a work-item could be an element of an array that is being processed. Work-items are bundled in work-groups like smaller subdivisions of an array, which allow for synchronization within the group. As such all work-items within a work-group are handled by the same Compute Unit (abbrv. CU), which for instance could be a CPU core.

    During execution multiple work-items are processed in parallel in so called wavefronts (AMD terminology) or warps (NVIDIA terminology). From here on the term wavefront will be utilized. Work-items within a wavefront are executed in lockstep, which means that all branches like both paths of if-conditions are run for very item, even when only a single work-item diverges. Therefore conditional constructs may lead to significant performance penalties.

    Wavefronts in modern AMD GPUs contain 64 work-items of which 16 are handled in parallel and the whole wavefront is executed within 4 clock cycles. Therefore work-groups should have a size multiple of 64 in order to evenly fill the wavefronts per CU. Having many wavefronts assigned per CU also enables OpenCL to hide latencies for required memory accesses. Whenever a wavefront needs to access memory, which would stall its threads, the CU switches to another wavefront and continues executing it until it also needs to access memory or finishes execution.

    \item [Memory Model]
    In its standard, OpenCL defines four different memory levels:
    \begin{description}[align=left]
      \item [private] Only accessible by a work-item itself.
      \item [local] Accessible by work-items within a work-group.
      \item [global] Can be read by and written to by the host as well as all work-items.
      \item [constant] Like global memory but read-only.
    \end{description}

    All of the available memory levels have different sizes, depending on the utilized device and also require different clock cycles per access. For example, reading data from local memory can be a magnitude faster than from global memory but global memory offers far more space and allows for synchronization among all work-groups.

\end{description}

OpenCL Kernels are written in OpenCL C, which is a superset of C99, containing additional keywords that are related to memory access and synchronization. Such Kernels have to be started from so called host code, which runs on the machine that contains the targeted devices. The host code is usually written in C or C++ as the standard offers APIs for these languages. Still, third parties offer bindings for other languages. The following list contains some available language bindings:

\begin{description}[align=left]
  \item [Java] JOCL, Aparapi
  \item [.NET] OpenCL.NET, Cloo
  \item [Python] PyOpenCL, PyCL
  \item [Ruby] Ruby-OpenCL
\end{description}

While the OpenCL standard enables the execution of identical code on heterogenous platforms, certain limitations still apply based on the capabilities of the utilized devices. For instance, code that uses the double data type can only be sucessfully run on devices that offer that feature. The following kernel code describes the addition of two arrays:

\begin{lstlisting}[language=C++]
#pragma OPENCL EXTENSION cl_khr_fp64 : enable

__kernel void run(__global double *a, __global double *b, __global double *c){
  int i = get_global_id(0);
  c[i]  = a[i] + b[i];
}
\end{lstlisting}

For this example the host code that is responsible for choosing an appropriate device, initiating data transfers and other auxiliary tasks has been omitted. With around 50 lines of code it is considerably longer than the actual kernel code. Executing the snippet on an early 2015 Macbook Air may lead to two different results. On the one hand, execution on the built-in i5-5250U CPU will correctly add the two provided arrays and save the result in a third array. On the other hand the included Iris Graphics 6100 GPU will produce an error as it does not support the required OpenCL FP64 extension.

While OpenCL supports more device types and vendors it seems questionable whether having a general programming model comes with the price of performance penalties when compared to the specialized CUDA platform. Karimi, Dickson and Hamze have shown that CUDA code may be ported to OpenCL with only few modifications but will run between 16\% and 67\% slower depending on the problem size of their implemented Monte Carlo simulation\cite{performance_comparison}. On the opposite, Fang and Varbanescu and Sips have come to the conclusion that such performance gaps are influenced by the differences between the programming models and may be prevalent due to programming mistakes and compiler optimizations\cite{comprehensive_performance_comparison}. They reason that with all issues in mind, OpenCL can reach the same performance as CUDA.

\section{Aparapi}
\label{aparapi}
Aparapi is a library that offers extensive functionality to ease the usage of the OpenCL API and minimize programming effort when developing OpenCL Kernels. Firstly, it contains bindings to access OpenCL functions through a Java Native Interface (abbrv. JNI), abstracting and bundling multiple low level calls to Java high level functions. Secondly, Aparapi is able to translate Java code to valid OpenCL kernels. Therefore, programmers can build classes that inherit the abstract Kernel class by Aparapi. The following code snippet shows such an example:

\begin{lstlisting}
final double[] a = new double[]{0, 1, 2, 3, 4, 5, 6, 7, 8, 9};
final double[] b = new double[]{0, 1, 2, 3, 4, 5, 6, 7, 8, 9};
final double[] c = new double[10];

Kernel kernel = new Kernel() {
  @Override
  public void run() {
     int i = getGlobalId();
     c[i] = a[i] + b[i];
  }
};

kernel.execute(10);

for(Double result:c){
  System.out.println(result);
}
\end{lstlisting}

The above code takes two arrays of doubles and computes the sum of contained elements at the same index, saving the result in a third array. Ultimately, each element from the result array is printed. It also highlights that input data may be defined in Java code itself and the results are again available in Java after execution. This is possible as Aparapi automatically copies the participating data back and forth between the host code and the executing device. For a more granular approach, developers can also define, which data should be written or read in order to improve performance. While an original OpenCL implementation would require roundabout 50 lines of auxiliary code, Aparapi abstracts tasks like device selection and data handling when no specific control is needed. This enables developers to program algorithms considerably faster and allows beginners quick access to OpenCL features without knowledge of low level mechanisms.

Although the code translation of Aparapi may work for many use cases, certain limitations apply to the Java code that can be handled. The following list covers some of these restrictions:

\begin{description}[style=nextline]
  \item [Data Types]
  The supported primitive data types are boolean, byte, short, int, long, and float. Additionally, only one-dimensional arrays comprised of these primitive types are allowed.

  \item [Object Orientation]
  While Aparapi supports Java objects, it can only translate classes whose attributes are again primitives. Also such classes should allow access to the attributes through getter and setter functions.

  \item [Control Flow]
  Aparapi only supports conditionals and counting for-loops. Switch, break, continue, enhanced for-loops and recursion are not allowed.

  \item [Specific Device Choice]
  Although it can be specified on which device type a kernel shall run, it is not possible to select a specific device for execution. For example, on a machine with four different GPUs one could not decide through Aparapi, which of the four GPUs should be used.
\end{description}
