\begin{center}\textsf{\textbf{Abstract}}\end{center}

\noindent OpenCL enables programmers to create parallel programs that can be executed on CPUs as well as GPUs. In combination with dOpenCL, OpenCL programs can be executed on remote machines via network. In this research the utilization of OpenCL and dOpenCL is abstracted behind the introduced framework called Dynamic OpenCL, which is written in Java. Dynamic OpenCL allows programmers to write distributed OpenCL programs in Java and adds sophisticated mechanisms for cluster management and dynamic resource adjustments. Dynamic OpenCL supports operation in various cluster setups such as local clusters that are connected to cloud services. It is shown that the framework can introduce significant performance benefits by distributing selected tasks with near perfect speedups. Even in heavily network limited cluster setups Dynamic OpenCL is able to increase performance noticeably by providing a network aware scheduling mechanism.


% <= Wenn die Arbeit auf Englisch verfasst wurde, verlangt das Studienreferat einen englischen UND deutschen Abstract

%\vspace{2cm}
%\selectlanguage{english}
%\begin{center}\textsf{\textbf{Abstract}}\end{center}
%Lorem ipsum dolor sit amet, consetetur sadipscing elitr, sed diam nonumy eirmod tempor invidunt ut labore et dolore magna aliquyam erat, sed diam voluptua. At vero eos et accusam et justo duo dolores et ea rebum. Stet clita kasd gubergren, no sea takimata sanctus est Lorem ipsum dolor sit amet. Lorem ipsum dolor sit amet, consetetur sadipscing elitr, sed diam nonumy eirmod tempor invidunt ut labore et dolore magna aliquyam erat, sed diam voluptua. At vero eos et accusam et justo duo dolores et ea rebum. Stet clita kasd gubergren, no sea takimata sanctus est Lorem ipsum dolor sit amet.
%\selectlanguage{ngerman}
