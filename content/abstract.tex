\begin{center}\textsf{\textbf{Abstract}}\end{center}

\noindent Creating parallelized software in order to reduce execution times is one of the main challenges in computer science. Writing parallel programs requires programmers to obtain extensive low level knowledge of the various available specifications and standards. Due to shortcomings of existing approaches, the goal of this research is to create a parallel computation framework that allows to distribute workloads among CPUs and GPUs within a cluster. The framework is expected to provide an abstract API that should be usable by programmers without deep knowledge of the underlying technologies. Additionally it should be possible to adjust computational resources dynamically depending on changing cluster utilizations.

Various parallelization methods like OpenMP, MPI and OpenCL are illustrated and discussed regarding their ability to contribute to the framework. Because of the possibility to produce portable code following a fixed programming model, OpenCL is chosen as the underlying base technology. In combination with dOpenCL, OpenCL programs can be executed on remote machines via network without changes to the original code. The resulting parallelization potential of OpenCL in conjunction with dOpenCL is abstracted behind the main contribution of this research -- Dynamic OpenCL. Dynamic OpenCL is written in Java and allows programmers to write distributed OpenCL programs in Java as well by employing Aparapi for code translations. It contains sophisticated mechanisms for cluster management and dynamic resource adjustments that enable its operation in various cluster setups such as hybrid clusters that are comprised of local and cloud resources. Depending on the respective cluster environment it is possible to adjust scheduling algorithms in order to gain performance improvements.

During an extensive evaluation it is shown that the framework can introduce significant performance benefits by distributing selected tasks with near perfect speedups. Even in heavily network limited cluster setups that incorporate cloud resources Dynamic OpenCL is able to increase performance noticeably by providing a network aware scheduling mechanism.

Based on Dynamic OpenCL a prototypical web server is built that allows users to submit computations through a user interface and enables them to speed up execution by booking additional cloud resources.

It is concluded that Dynamic OpenCL is able to provide speedups in various cluster environments, mainly depending on the nature of submitted workloads and available network bandwidths to remote devices. With further improvements that aim at reducing the impact of limited network bandwidths, the overall performance might be improved even further in the future.

% <= Wenn die Arbeit auf Englisch verfasst wurde, verlangt das Studienreferat einen englischen UND deutschen Abstract

%\vspace{2cm}
%\selectlanguage{english}
%\begin{center}\textsf{\textbf{Abstract}}\end{center}
%Lorem ipsum dolor sit amet, consetetur sadipscing elitr, sed diam nonumy eirmod tempor invidunt ut labore et dolore magna aliquyam erat, sed diam voluptua. At vero eos et accusam et justo duo dolores et ea rebum. Stet clita kasd gubergren, no sea takimata sanctus est Lorem ipsum dolor sit amet. Lorem ipsum dolor sit amet, consetetur sadipscing elitr, sed diam nonumy eirmod tempor invidunt ut labore et dolore magna aliquyam erat, sed diam voluptua. At vero eos et accusam et justo duo dolores et ea rebum. Stet clita kasd gubergren, no sea takimata sanctus est Lorem ipsum dolor sit amet.
%\selectlanguage{ngerman}
